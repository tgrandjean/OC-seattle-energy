%----------------------------------------------------------------------------------------
%	ARTICLE CONTENTS
%----------------------------------------------------------------------------------------

\section{Introduction}


%------------------------------------------------

\section{Données disponibles}

\subsection{généralités}

Les données sont disponibles gratuitement sur le site de la ville de seattle et
également sur la plateforme "Kaggle".

Le jeu de données contient deux fichiers :
\begin{itemize}
  \item{2015-building-energy-benchmarking.csv}
  \item{2016-building-energy-benchmarking.csv}
\end{itemize}

Les deux jeux de données contiennent essentiellement les mêmes variables mais
certaines (Par exemple : la localisation) sont encodées différemment.

\subsection{Notes}

Les seules différences notables entre les deux jeux de données sont:
\begin{itemize}
  \item{Le jeu de 2016 contient 36 observations de plus que celui de 2015.}
  \item{Les données de 2016 n'ont pas la variable "OtherFuelUse"}
  \item{Les données de 2015 contiennent une variable "Location" contenant les
       informations suivantes}
  \begin{itemize}
    \item{City}
    \item{State}
    \item{ZipCode}
    \item{Address}
    \item{Latitude et longitude}
  \end{itemize}
  \item{Les colonnes suivantes ont été renommées}
  \begin{itemize}
    \item{Comment : comments}
    \item{GHGEmissions(MetricTonsCO2e) : TotalGHGEmissions}
    \item{GHGEmissionsIntensity(kgCO2e/ft2) : GHGEmissionsIntensity}
  \end{itemize}
\end{itemize}

%------------------------------------------------

\section{Analyse exploratoire des données}

\subsection{Travail préliminaire sur les données}

Les deux jeux de données sont uniformisés au niveau des variables
(la colonne "Location" du jeu de données de 2015 est transformée en colonnes
"Address", "ZipCode", "Latitude", "Longitude", ...) puis aggrégés pour former
un jeu unique contenant 47 variables et 6716 observations.

On a donc les variables suivantes :
  voir table \ref{tab:table-variables}

\begin{table}[]
  \caption{liste des variables}
  \label{tab:table-variables}
\begin{tabular}{ll}
\toprule
{} &         0 \\
\midrule
OSEBuildingID                   &     int64 \\
BuildingType                    &  category \\
PrimaryPropertyType             &  category \\
PropertyName                    &    object \\
TaxParcelIdentificationNumber   &    object \\
CouncilDistrictCode             &     int64 \\
Neighborhood                    &  category \\
YearBuilt                       &     int64 \\
NumberofBuildings               &   float64 \\
NumberofFloors                  &   float64 \\
PropertyGFATotal                &     int64 \\
PropertyGFAParking              &     int64 \\
PropertyGFABuilding(s)          &     int64 \\
ListOfAllPropertyUseTypes       &    object \\
LargestPropertyUseType          &  category \\
LargestPropertyUseTypeGFA       &   float64 \\
SecondLargestPropertyUseType    &  category \\
SecondLargestPropertyUseTypeGFA &   float64 \\
ThirdLargestPropertyUseType     &  category \\
ThirdLargestPropertyUseTypeGFA  &   float64 \\
YearsENERGYSTARCertified        &    object \\
ENERGYSTARScore                 &   float64 \\
SiteEUI(kBtu/sf)                &   float64 \\
SiteEUIWN(kBtu/sf)              &   float64 \\
SourceEUI(kBtu/sf)              &   float64 \\
SourceEUIWN(kBtu/sf)            &   float64 \\
SiteEnergyUse(kBtu)             &   float64 \\
SiteEnergyUseWN(kBtu)           &   float64 \\
SteamUse(kBtu)                  &   float64 \\
Electricity(kWh)                &   float64 \\
Electricity(kBtu)               &   float64 \\
NaturalGas(therms)              &   float64 \\
NaturalGas(kBtu)                &   float64 \\
OtherFuelUse(kBtu)              &   float64 \\
TotalGHGEmissions               &   float64 \\
GHGEmissionsIntensity           &   float64 \\
DefaultData                     &    object \\
Comments                        &    object \\
ComplianceStatus                &    object \\
Outlier                         &    object \\
Latitude                        &   float64 \\
Longitude                       &   float64 \\
Address                         &    object \\
State                           &    object \\
City                            &    object \\
ZipCode                         &    object \\
\bottomrule
\end{tabular}

\end{table}

\subsection{Analyse des variables}
Un rapport d'analyse exploratoire a été généré à l'aide de la bibliothèque pandas-profiling.
Le rapport est disponible à l'adresse :
\href{https://tgrandjean-oc-reports.s3.eu-west-3.amazonaws.com/seattle-energy/profiling_report.html}{AWS S3 bucket}
%------------------------------------------------

\section{Modèle}

\subsection{Données d'entrée (input)}

\subsection{Données de sortie (output)}

\subsection{Algorithme utilisé}

\subsection{Entrainement du modèle}


%------------------------------------------------

\section{Résultats}


%------------------------------------------------

\section{Conclusion}


%------------------------------------------------
